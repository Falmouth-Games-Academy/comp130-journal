% Please do not change the document class
\documentclass{scrartcl}

% Please do not change these packages
\usepackage[hidelinks]{hyperref}
\usepackage[none]{hyphenat}
\usepackage{setspace}
\doublespace

% You may add additional packages here
\usepackage{amsmath}
\usepackage{graphicx}
\usepackage{url}
\usepackage[autostyle]{csquotes}

% Please include a clear, concise, and descriptive title
\title
{Research Journal}

% Please do not change the subtitle
\subtitle{COMP130}

% Please put your student number in the author field
\author{Student number: 1607539}

\begin{document}

\maketitle

\blockquote {A journey of a thousand miles begins with a single step - Lao Tzu}\\
\indent\blockquote {On second thoughts, let's not go to Camelot. 'Tis a silly place - Graham Chapman}%{All the roads we have to walk are winding - Noel Gallagher}{I got the right street, but the wrong direction - Rufus, Chaka Khan}

\section*{\underline{Pathfinding in games}}

Pathfinding stands out as one of the most common applications of AI research to games \cite{botea2013pathfinding}.  In practice the process must be as efficient as possible as there many are many other demands on the CPU, such as physics and graphics \cite{sturtevant2007memory}.  For the optimum solution, A* is known as one of the best search algorithms \cite{algfoor2015comprehensive}.

\subsection*{A*}

By extending Dijkstra's algorithm to use heuristics relating to an end node (in Dijkstra's algorithm H is always 0) \cite{lester2005pathfinding}, A* is the \textit{de facto} pathfinding algorithm \cite{leigh2007using}, and with an admissible heuristic, returns optimal solutions.  Dijkstra's algorithm was the only choice for pathfinding until the last two decades \cite{cui2011direction}, and should still be used when there is no end node, for example, were there several targets and an actor were needed to move to the closest \cite{lester2005pathfinding}.

\subsection*{Heuristics}

The Oxford dictionary defines Heuristics in regards to computing as proceeding to a solution by trial and error or by rules that are only loosely defined.  A* uses a weighting for each tile defined as the sum of the distance from the start node, and the distance from the target as a heuristic (H), to find the path of least cost. Commonly written as finding the smallest value of $f(n) = g(n) + h(n)$.  

The Manhattan distance is a well known admissible heuristic, and is defined as $\Delta x + \Delta y$ \cite{botea2013pathfinding} which can be seen on an obstacle free grid as the direct route and distance.  The octile distance is a variant of the Manhattan distance adapted to 8-connected maps. It is defined as $\sqrt{2} \cdot m + (M - m)$, where $m = min(\Delta x, \Delta y)$ and $M = max(\Delta x, \Delta y)$ \cite{botea2013pathfinding}, essentially examining the hypotenuse of an equilateral right angled triangle with sides of length $m$.  $\sqrt{2}$ features prominently in square grid pathfinding although it is often approximated.  The degree of accuracy should be highly pertinent, although for short paths, $\sqrt{2} \approx 1.4$ can be sufficiently accurate.

\subsection*{Optimizations}

Given the success and popularity of A* pathfinding, there has been considerable effort to improve and refine it.  Typical speed-up enhancements include reducing the size of the search space using abstraction, and building more informed heuristics \cite{harabor2010breaking}, although the execution time, memory overhead, and whether the environment of the search system is static, dynamic, or real-time \cite{hart1968formal} (working under the assumption that there is a limited amount of time to make a move) must be considered.  Specific per case optimization is most important. For instance, a slightly overestimating heuristic increases the speed at the expense of accuracy \cite{millington2016artificial}, or were the map not to feature dead ends that would require back tracking then using the node with the best heuristic as far as a goal node is favourable \cite{togelius20102009}.

Dynamically changing environments \cite{botea2011ultra} are a common occurrence in games and A* has difficulty in solving the problem of random dynamics obstacles avoidance \cite{wang2012game}.  Sometimes we just want to start the movement in the correct direction before its ultimate outcome is known, as replanning from scratch every time the graph changes can be very computationally expensive \cite{ferguson2005guide}.  Pathfinding in a smaller space abstraction can reduce the memory footprint \cite{lawrence2013database} especially when the end node of the solution is subject to change.  A lookahead search is also effective for such problems \cite{korf1990real}, but although looking ahead can improve the solution quality, suboptimal solutions can be produced even with admissible and consistent heuristics \cite{bulitko2003lookahead}.

When maintaining the open nodes of an A* search, it is better to use an array of stacks than a priority queue. If $N$ is the number of elements in the priority queue, the insertion and the extraction of an element both take $O(log N)$ which is faster than a constant time for inserting and a linear time for extracting as in the list implementation \cite{cazenave2006optimizations}.  Correctly maintaining the priority queue is key for optimization, allowing a lazier initialisation for the node cache; it is more efficient to use an array of the size of the map as the game map already fits in memory \cite{cazenave2006optimizations}.

\subsection*{Variations}

Common optimisations are to implement hierarchy (HPA*) with space abstractions or by using a navmesh \cite{cui2011based}.  HPA* breaks down the map into sections and is significantly faster than low level A* \cite{botea2004near}, speeding up a search by replacing it with a series of smaller searches \cite{botea2013pathfinding} gradually refined into a low level solution.  It is also possible to build more than one level of abstraction for HPA* \cite {jansen2007hpa}. By precomputing ideal routes across these larger sections; encouraging the algorithm to expand nodes from the perimeter rather than the interior \cite{harabor2012fast}, or node pruning with jump points \cite{harabor2011online}, symmetries (identical length paths where the only difference between them is the order in which moves appear \cite{harabor2012fast}) can be drastically reduced, trading off the real time CPU resources with optimality of the solution. 

Iterative A* is a space-efficient version of A*, but it suffers from cycles in the search space (the price for using no storage), and repeated visits to states (the overhead of iterative deepening) \cite{bjornsson2005fringe}.

Block A* pre-computes a database of all possible block topologies.  For each block, optimal distances between pairs of boundary cells are stored.  At runtime, Block A* expands an entire block at a time, as opposed to one individual cell \cite{botea2013pathfinding}

\subsection*{Always with the triangles, your solution to everything is triangles}

The most common method for building navigation meshes is by vertex-based decomposition \cite{hale2009full} collapsing corridors and tree components \cite{demyen2006efficient}, essentially splitting the map into triangles.  This has the advantage that the obtained triangular cell decomposition of the environment has $O(n)$ cells, where $n$ is the number of segments used to describe the obstacles in the environment \cite{kallmann2010navigation}.  In fact, a triangulation A* variant is being used in the StarCraft 2 game engine \cite{ontanon2013survey}.

\subsection*{Scrubbing}

Due to dead ends and corners, or when the initial heuristic values are overly optimistic \cite{bulitko2011real}, agents can ``scrub'' (repeatedly revisit) the state space.  State revisitation is inherently suboptimal, and can look unnatural which can reduce player immersion.  It can be reduced by setting sub goals (temporary end nodes) in the state abstractions \cite{bulitko2010case}.

A hill climbing agent performs no heuristic updates \cite{huntley2013search} and can brute force its way past scrubbing behaviour by only examining its neighbours.  Although it must be used with a cut off number of steps to avoid unnatural behaviour, as it is vulnerable to local plateaus \cite{richards1992learning}.

\bibliographystyle{ieeetran}
\bibliography{references}

\end{document}


\subsection*{WIP}

scrubbing? Hill climbing moves?

Monte carlo

Link Ed's stuff

iteritive

Cellula autonoma

A* whilst thorough, can be expensive


Partial Refinement A* (PRA*) combines hierarchical abstraction with fast partial path refinement to effectively interleave path planning with path execution \cite{botea2013pathfinding} mapping small connected regions, to tiles on the next level of abstraction while preserving connectivity. Creating pyramids of coarser and coarser map representations up to the point in which only single tiles remain that represent the original connected components. \cite{botea2013pathfinding}

For finding paths subject to given time and path quality constraints, we choose an abstraction layer and find nodes representing the start and endpoints on the original map. Starting at this level, PRA*(k) uses A* to compute the shortest path and then projects the first k steps down to the next map abstraction level, where the shortest path is determined by only considering a small corridor around the projected path, etc. This process quickly finds high-quality initial paths on the original map which allows objects to start moving in the right direction quickly. \cite{botea2013pathfinding}

HAA* [12] extends HPA* in two directions. First off, in classical pathfinding, a map is partinioned into traversable terrain and blocked areas. HAA* makes a finer distinction, allowing to define several types of terrain, such as ground, water, and walls. Secondly, HAA* handles mobile units with variable sizes and variable terrain traversal capabilities. \cite{botea2013pathfinding}


The general solving framework includes a planning step, a learning step and an acting step. In the planning step, the agent explores the area around its current location.
