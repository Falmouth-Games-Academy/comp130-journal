% Please do not change the document class
\documentclass{scrartcl}

% Please do not change these packages
\usepackage[hidelinks]{hyperref}
\usepackage[none]{hyphenat}
\usepackage{setspace}
\usepackage{graphicx}
\doublespace

% You may add additional packages here
\usepackage{amsmath}

% Please include a clear, concise, and descriptive title
\title{Can Video games be tailored to their User by procedurally creating challenging emergent gameplay and narrative experiences.}

% Please do not change the subtitle
\subtitle{COMP130 - Research Journal}

% Please put your student number in the author field
\author{1507729}

\begin{document}

\maketitle

\abstract{}

\section{Can Users Trust Computer Agents?}

It's very common in video games to see a computer agent or non player character (NPC). NPC characters are in computer terms agents which will perform certain actions depending on predetermined rule sets, a simple example might be if an NPC can see the player he will attack them. However if you wished to build a procedural narrative experience it's important that NPC characters who are relevant to your plot lines are capable of more complex or varied actions. An upcoming example is the Middle-earth: Shadow of war which features procedurally created NPC characters with unique behaviours that allow users to create their own narrative experiences. \cite{monolith2017shadow}. It's worth noting that various studies have looked into creating a sense of trust with NPC characters and looking at ways in which users interact with NPC's. \cite{do2016trust} On the opposite end of this spectrum it's also worth noting ways in which NPC's or game agents could decieve or lie to a player. Once such example of this research is shown here \cite{cowling2015emergent}.

\section{How Creative can Creative Computing be for Narrative Design?}

The study of social computing and computation creativity are still relatively new however the ideas of narrative creation were perhaps first realised with Kleins Novel Writer \cite{klein1973automatic} back in 1973 and then in many works since \cite{gervas2009storytelling}. Kleins system looked at creating murder mystery novels from a number of predefined options. This however is still limited and the structure will remain the same across all stories. 


\bibliographystyle{ieeetran}
\bibliography{references}

\end{document}
