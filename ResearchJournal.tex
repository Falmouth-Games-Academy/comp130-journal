% Please do not change the document class
\documentclass{scrartcl}

% Please do not change these packages
\usepackage[hidelinks]{hyperref}
\usepackage[none]{hyphenat}
\usepackage{setspace}
\usepackage{graphicx}
\doublespace

% You may add additional packages here
\usepackage{amsmath}

% Please include a clear, concise, and descriptive title
\title{Can Video games be tailored to their User by procedurally creating challenging emergent gameplay and narrative experiences.}

% Please do not change the subtitle
\subtitle{COMP130 - Research Journal}

% Please put your student number in the author field
\author{1507729}

\begin{document}

\maketitle

\abstract{Video games are a part of everyday life for many people, from phone games to fully blown console and PC releases, with video games becoming so prevalent in so many lives and to so many people creating new or interesting content can become increasingly challenging. This paper will be exploring ways in which existing technology can be used to create new and interesting characters, narratives and adapting to fit their users.}

\section{Can Users Trust Computer Agents?}

One of the most common elements in video games are computer agent or non player character (NPC). NPC characters are in computer terms agents which will perform certain actions depending on predetermined rule sets, a simple example might be if an NPC can see the player he will attack them. However if you wished to build a procedural narrative experience it's important that NPC characters who are relevant to your plot lines are capable of more complex or varied actions. An upcoming example is the Middle-earth: Shadow of war which features procedurally created NPC characters with unique behaviours that allow users to create their own narrative experiences. \cite{monolith2017shadow}. It's worth noting that various studies have looked into creating a sense of trust with NPC characters and looking at ways in which users interact with NPC's. \cite{do2016trust} \cite{granatyr2015trust} On the opposite end of this spectrum it's also worth noting ways in which NPC's or game agents could deceive or lie to a player, by limiting the knowledge of what the NPC is aware of it's possible to get a different response. Once such example of this research is shown here \cite{cowling2015emergent}. Though the applications for such a system would be dependant on context, one such example could be a merchant NPC who tells players the selling price for their items is much lower than the real average. This could be an effect way of making NPC's less omniscient and more human-like.

\section{Procedural Content Generation in Games}

Procedural content generation (PCG) is growing in popularity in the realm of video game creation as a tool and as a feature which can create seemingly infinite new variations of pre-existing assets and in many cases create new content as well, this can be from creating whole ecosystems to urban environments and beyond. \cite{hendrikx2013procedural} Their are many games which are built around creating content for users at launch one very popular example would be the Worms series, which uses PCG to create new and interesting levels. \cite{team1995worms} Though many other examples of this exist and this type of generation has only become more popular in recent years, showing up in AAA and indie games alike. \cite{ubisoft2008farcry} \cite{mojang2011minecraft} Overtime this technology has lead to lots of research for new ways to design games and other technology to best utilise PCG \cite{smith2012pcg}, as you want to make unique environments however you also need to make sure that the PCG content is enhancing your product not subtracting from it. This type of PCG is can be referred to as Experience-Driven Procedural Content Generation (EDPCG) \cite{craveirinha2015designing}


\section{How Creative can Computational Creativity be for Narrative Design?}

The study of social computing and computation creativity is still relatively new however the ideas of narrative creation were perhaps first realised with Kleins Novel Writer \cite{klein1973automatic} back in 1973 and then in many works since \cite{gervas2009storytelling}. Kleins system looked at creating murder mystery novels from a number of predefined options. This however is still limited and the structure will remain the same across all stories. A more advanced machine would be BRUTUS, \cite{bringsjord2000artificial} which allows for stories which can create intrigue and mystery. If video games were to incorporate this kind of technology architecture in a way that the user was able to interact with new narratives and further develop existing stories, it has the potential to create a sustained and engaging experience.

\section{A Reactive System For Players}

Taking the time to look at analytical data can greatly inform future works of a business or team. However gathering and sorting this data can be difficult at times, especially if you have a large user base. You also can run into the problem of each user having a different personality and wanting to get a different experience than other users.

Through the use of in-game analytics it's possible to dynamically monitor and change certain aspects of a game, much like how the Game Master in traditional table-top role playing games might change encounters or situations based on how their party approaches certain situations making them easier or more difficult. \cite{sega1999house} In fact a very good example of this is "The AI Director" from the popular co-op shooter Left 4 Dead \cite{valve2008left}. 

Lots of research has taken place looking for ways to monitor users especially in the field of Serious Games, or games for education as it can also be known. \cite{shute2009melding} \cite{hauge2014implications} This is called Learning Analytics \cite{fournier2011value} and from an educational stand point being able to not only quantify this data but also instantiate changes based upon them.

One example of this type of dynamic alteration to gameplay can be found in most typical endless runner games, which scale their difficulty to make more enemies who are faster as time progresses.Though one clear example of this is the hamlet system embedded in the half-life engine \cite{hunicke2005case}.  Although not common, it isn't unheard of for games to introduce features allowing users to rate their experiences to provide feedback to developers to make future improvements \cite{ubisoft2013assassins}. If this type of system was added into a game which used a form of procedural creation it could through weighted percentages create more encounters of a similar nature allowing the player to experience the particular aspect of gameplay or narrative experience they enjoyed, more frequently.

\section{Creating Systems for Autonomous Creation}

Technology is always advancing and although currently computers are limited the idea of autonomous systems capable of updating themselves independently and being able to create solutions to problems, is called Autonomic Computing \cite{viroli2016combining} \cite{capodieci2013designing} \cite{krupitzer2015towards} and you could say it is the parent to creative computing. 

Video games obviously can't be completely autonomous as then they wouldn't be games, but by creating a more symbiotic system /cite{kephart2015symbiotic} that creates and learns from a human user it's possible to create a dynamic and personal experiences, though this area of research is still new and not likely to be available for mainstream or commercial usage for sometime. 

"A concept of symbiotic computing was proposed as a computing methodology to compute humanity and sociality in order to bridge the gap between the Real Space (RS) and the Digital Space (DS)" \cite{sugawara2008design} The idea of helping to bridge the gap between digital and real space could be looked at from the perspective of immersion and creating a sense of immersion, one such intention for symbiotic systems is to create cognitive agents who can recognise actions and intentions of users. The idea of symbiotic computing seems to stem from the idea of creating agents able to adapt, react and predict outcomes from users. 

Looking at Mixed-initiative content creation, which is a form of symbiotic computing \cite{liapis2016mixed} \cite{yannakakis2014mixed}. Allowing users to have content created within parameters they can adjust and control. This allows users to have more control over how they want to experience the game. This is just the tip of the iceberg however as it can be more than just a tool for users to create content for themselves but also a way in which developers can prototype ideas and also create content quickly and easily by selecting a computerized model and then adapting it iteratively from there.

\section{Conclusion}

From the above research it certainly seems possible that video games could be creating content that uses multiple systems to monitor, react and create content for players. However  as this is still an area of research it's unlikely that we will be seeing any AAA games made from this technology, and due to the complexity of the systems it's also unlikely that an indie studio will be combining these systems as the technical expertise required is seeming quite high. In the future it's more than likely however that we could see a surge of games with these kind of features, in the same way that procedural content generation in video games has seen a surge of interest. 

\bibliographystyle{ieeetran}
\bibliography{references}

\end{document}
